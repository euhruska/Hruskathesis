% star NOT needed to make sure that the 'Conclusions' are  included in chapter numbering
\chapter{Conclusions}
%
Coarse-graining collectively denotes a variety of methods that are used to build eective
surrogate models of a given physical system; for example, classical thermodynamics is a
surrogate model of classical dynamics of many-particle systems, which can be obtained by
resorting to statistical mechanics.
Many systems display complex hierarchy of aggregation levels and, at a larger scale,
the truly physically meaningful observables are those quantities that faithfully characterize
the collective properties. In principle, the nature laws at high resolution could be scaled
up and would still describe the physics at a lower resolution; however, as the number of
degrees of freedom increases exponentially, the “big picture” properties are washed out by
an overwhelming and highly dimensional description. For instance, recording the position
and velocity of each ideal gas pld be the most detailed
description we could ever have, but just looking at such a dataset as is, would not tell us
anything about the gas temperature and pressure.
This suggests that the level of detail of a physical model needs to be modulated according
to the resolution at which the system is investigated: in particular, the high resolution
degrees of freedom should be decimated and replaced by a (smaller) number of eective
new ones and a set of interactions, and coarse-graining methods investigate how that can
be accomplished in practice.
In this Dissertation we discussed a number of topics associated with formulating a coarsegrained
model of an all-atom representation, by adopting a bottom-up approach. Extensive
all-atom equilibrium simulation data are available nowadays, and already encode the low
resolution description, e.g. slow timescale dynamics, perturbed by the high resolution de
tails, e.g. the (ultra) fast vibrations of the individual atoms. It is then quite natural to
adopt a data-driven stance, and use physics-based machine-learning techniques to extract
the essential information while filtering out the fast motions.
We first discussed how the essential information in the biomolecular context is fully
captured by the dominant eigenvectors and eigenvalues of the generator underlying the
dynamics. Then, an optimized technique named Variationally Optimized Diusion Maps
was introduced (Ch. 3) as a self-consistent unitary analysis framework to estimate eigenvectors
and eigenvalues from an all-atom simulation. The method was applied to a simple
protein, performs better than other popular approaches, and is recommended as a rigorous
validation technique for comparing dierent calculation setups and parameter choices.
From a strictly algorithmic point of view, the method we propose requires the computation
and storage of large (sparse) correlation matrices, which is greatly time-consuming
and memory expensive. A compressed sensing method named oASIS was discussed (Ch. 4)
which allows to efficiently reconstruct the full matrices by using just a small subset of their
entries.
Ch. 5 discusses how the collective coordinate formulation can be adapted to identifying
a set of coarse-grained degrees of freedom. The Structure and State Space Decomposition
method extracts dynamically coherent groups of atom by post-processing an allatom
simulation, and those sets are good candidates for coarse-grained “beads”. Applying
the method to two proteins revealed the intriguing possibility that a kinetically consistent
coarse-graining description may require a dierent level of resolution according to the region
of the configuration space being visited: indeed, the beads are shown to merge together
dierently in dierent metastable states.
A set of coarse-grained degrees of freedom needs to be accompanied by their eective
equations of motions. It is discussed in Ch. 6 and 7 how two complementary strategies
(Sparse Learning of Stochastic Dynamics and Spectral Coarse-Graining) are successfully
applied to toy-models to agnostically recover appropriate equations of motion. Both strate


gies start out by choosing a parametric Ansatz for the terms in a diusive equation, and
then a fitting problem on the data is formulated: the solution to the problem is the optimal
dynamical model which best interpolates the data.
The novel approaches discussed in this Dissertation all contribute to shed light onto a
more rigorous and parameter-free coarse-graining stance. This is a substantial step into the
right direction, but there are still a number of crucial issues that require more attention.
First of all, all methods require an extensive all-atom equilibrium trajectory as an input,
which might be challenging for large scale systems. In order to mitigate its impact,
we have been actively working on implementing transferability of the models, especially
as far as identifying coarse-grained degrees of freedom is concerned. Transferability is the
quintessential coarse-graining issue, which would ensure protocol systematicity and reproducibility
across dierent systems. In order to target that, the Structure and State Space
Decomposition is being run on a vast pool of dierent proteins, and the dierent decompositions
of the primary sequence are compared in order to identify common decomposition
patterns, (e.g., Which atomic species stay together? Do the backbone and side chains move
coherently?) The hope is to generate a comprehensive database with a set of recipes for
lumping together atoms into eective beads across dierent systems.
Second, even if the decomposition were reproducible, it is not clear yet which coordinates
shall be used to describe the eective beads. This is a crucial point, since this choice will
aect the functional form of the eective equations of motion. A viable strategy is to model
each bead as an ellipsoid and use the coordinates of its center of gravity as eective ones.
Current explorations are broadly investigating dierent aspects associated with this issue.
Third, there is no guarantee that the coarse-grained equations of motions will have a
closed form. So far, in our examples, choices were made which ensured that the resulted




that cannot be written out in a closed form (Mori-Zwanzig formalism). The most general
result available is that the projected dynamics in the space of the dominant eigenvectors
would still be a Markovian (non-homogeneous) diusion process: however, working in such a
reduced configurational space would compromise the physical interpretability of the model;
we do know that the eigenvectors do not have a structural interpretation. Implementing
dierent levels of approximation into the Mori-Zwanzig projection formalism, and compare
the resulting dynamical models is a type of research that will need to be addressed, in order
to broaden the number of dynamical models whose parameters can be learned.
All in all, the work detailed in this Dissertation is a considerable leap forward in the
interdisciplinary field of data-driven molecular coarse-graining, and presents a unitary vision
of how configurational and structural coarse-graining scan be tackled simultaneously. This is
not of course the end of the story, and represents a new starting point for more investigations.
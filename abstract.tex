\afterpage{\null\newpage}
\thispagestyle{empty}
\begin{abstract}

The challenge of predicting the conformation dynamics of biomolecules is at the core of our limited ability to understand many biophysical processes. This includes many questions around the causes of many diseases and biophysics theory. \emph{Adaptive sampling} is an approach to increase our ability to predict conformational dynamics. 

The challenge of effectively sampling the conformation dynamics of biomolecules is one example of a more general problem.  In terms of physics, the general problem is the accurate sampling of the time-dynamics of high-dimensional stochastic systems. The high-dimensionality combined with a complex energy landscape prevents any simple solutions. 

Many approaches to unravel this challenge have in the past achieved significant improvements. In the case of proteins, the timescales where we are able to predict the conformational dynamics increased by many orders of magnitudes to the millisecond scale. This illustrates the magnitude of the challenge since the current state-of-art can only predict the accurate behavior for small proteins. For most of the larger biomolecules, we are not able to predict the accurate behavior. This is not only caused by the several magnitude longer timescales but also an order of magnitude larger sizes of the biomolecules.  

Considering the broad scope of the general challenge for sampling a high-dimensional stochastic system, this Dissertation will focus only on improving the prediction of conformational dynamics of proteins. This restriction doesn't limit the applicability of the methods investigated to other high-dimensional stochastic systems.

First, the prediction of the effectivity of different adaptive sampling strategies will be discussed. Due to significant stochasticity and protein-to-protein variation, the choice of adaptive sampling strategy is not apparent. The performance of different goals varies as well. 

Second, to deepen our theoretical understanding of adaptive sampling strategies, an upper limit for the performance of any adaptive sampling strategy is developed. This theoretical upper limit allows us to understand the potential and limits of adaptive sampling.

Third, adaptive sampling is heavily dependent on software due to the thousands or millions of individual steps necessary to be performed, all to be executed in an efficient fashion on a High-Performance Computer (HPC). Here we show the development of the software package \emph{ExTASY}. This framework allows executing all the necessary steps for adaptive sampling while reducing the work-load. The innovations of ExTASY are both the high-scalability and the modularity, which allow for an easy change of the adaptive sampling strategies and maintenance. 

Finally, the package \emph{ExTASY} will be applied to show the results of adaptive sampling for several proteins. Future developments to extend the investigated approaches to longer timescales will be addressed.

\end{abstract}



\afterpage{\null\newpage}
\thispagestyle{empty}
\begin{abstract}

The challenge of predicting the conformation dynamics of biomolecules is at the core of our limited ability to understand many biophysical processes. This includes many questions around the causes of many diseases and biophysics theory. \emph{Adaptive sampling} is an approach to increase our ability to predict conformational dynamics. 

The challenge of effectively sampling the conformation dynamics of biomolecules is one example of a more general problem.  In terms of physics, the general problem is the accurate sampling of time-dynamics of high-dimensional stochastic systems. The high-dimensionality presents here several challenges. The high-dimensionality with a complex energy landscape prevents any simple solutions. Additionally the. 
dimensional curse. 

In the past many approaches to unravel this challenge have in the past achieved significant improvements. In the case of proteins, the timescales where we are able to predict the conformational dynamics increased by many orders of magnitudes to the millisecond scale. This illustrated the magnitude of the challenge since the current state-of-art can only small proteins and for most, larger biomolecules we are not able to predict the accurate behavior. This is not only caused by the several magnitude longer timescales but also order of magnitude larger sizes of the biomolecules.  

Considering the broad scope of the general challenge for high-dimensional stochastic system this Dissertation will focus on only improving the prediction of conformational dynamics of proteins. This restriction doesn't limit the applicability of the methods investigated to other high-dimensional stochastic systems.

First the prediction of the effectivity of different adaptive sampling strategies will be discussed. Due to significant stochasticity and protein-to-protein variation the choice of adaptive sampling strategy is not obvious. The performance for different goals varies as well. 

Second, to deepen our theoretical undertanding adaptive sampling strategies, a upper limit for the performance of any adaptive sampling strategy is developed. This theoretical upper limit allows us to g In particular 

Third, adaptive sampling is heavily dependent on software due to the thousands or millions individual steps necessary to be performed, all in an effecient fashion of a High-Performance Computer (HPC). Here we show the development of the software package \emph{ExTASY}. This framework allows to execute all the necessary steps for adaptive sampling while reducing the work-load. The innovations with ExTASY are both the high-scalability and the modularity which allow for an easy change of the adaptive sampling strategies and maintanance. 

Future developments to extend the investigated approaches to longer timescales will be addressed.

\end{abstract}



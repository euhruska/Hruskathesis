\afterpage{\null\newpage}
\thispagestyle{empty}
\begin{abstract}

At the core of our limited ability to understand many biophysical processes is the challenge of predicting the conformational dynamics of biomolecules. This challenge includes many open questions around the biophysical causes of many diseases or open questions in biophysics theory. \emph{Adaptive sampling} is an approach to increase our ability to predict conformational dynamics. Adaptive sampling is a class of sampling strategies, where an ensemble of molecular dynamics trajectories is generated, where the starting points for the individual trajectories depend on the previously simulated trajectories. This approach will be investigated in this thesis. 

The application of adaptive sampling to biomolecules is one example of the more general problem of accurately sampling the time-dynamics of high-dimensional stochastic systems. The high-dimensionality, combined with a complex energy landscape, impede simpler approaches. Due to the broad scope of the general challenge, this Dissertation will focus only on improving the prediction of conformational dynamics for proteins. 

Many previous approaches to unravel this challenge have achieved significant improvements. In the case of proteins, the timescales where we can predict the conformational dynamics have increased by many orders of magnitudes to the millisecond scale. Despite the improvements, the current state-of-art can only predict the accurate behavior for small proteins. This illustrates the magnitude of the challenge. For most of the larger biomolecules, we are not able to simulate the precise behavior. This is not only caused by the several magnitudes longer timescales for these larger systems but also an order of magnitude larger sizes of these biomolecules.  

In this thesis, the adaptive sampling of conformational dynamics will be investigated in several steps. First, the prediction of the effectivity of different adaptive sampling strategies will be discussed. Due to significant stochasticity and protein-to-protein variation, the choice of adaptive sampling strategy is not apparent. The performance of different strategies for different goals varies as well. 

Second, to deepen our theoretical understanding of adaptive sampling strategies, an upper limit for the performance of any adaptive sampling strategy is developed. This theoretical upper limit allows us to understand the potential and limits of adaptive sampling.

Third, adaptive sampling is heavily dependent on software due to the necessary thousands or millions of individual steps. All these steps have to be executed efficiently on a High-Performance Computer (HPC). Here we show the development of the software package \emph{ExTASY}. This framework allows performing all the necessary steps in adaptive sampling while reducing the workload. The innovations of \emph{ExTASY} are both the high-scalability and the modularity. The modularity allows for an easy change of the adaptive sampling strategies and better maintainability. \emph{ExTASY} is reducing the entry barrier to utilizing adaptive sampling.

Finally, the package \emph{ExTASY} will be applied to show the results of adaptive sampling for several proteins. Future developments to extend the investigated approaches to longer timescales will be addressed.

All the approaches mentioned above facilitate further advancements in predicting conformational dynamics of larger biomolecules.

\end{abstract}



\thispagestyle{empty}
\begin{abstract}

\emph{Coarse-graining} \RED{2 pagesis an ubiquitous concept in the sciences, and
denotes a variety of diverse methods to consistently formulate a low
resolution model of a physical system. If detailed data from a higher-resolution
model is available, a popular \emph{bottom-up} approach consists in
\textit{renormalizing} that information into a surrogate model, by properly
filtering out \emph{non-essential} details, while preserving what
is considered\emph{ essential}. }

For biological molecules, a coarse-grained model requires groups
of atoms to be replaced by effective degrees of freedom and their
new interactions to be specified. In addition, the long timescale
features of the original dynamics shall be preserved, since they correlate
with physico-chemically relevant conformational rearrangements, such
as (mis)folding. It can be shown that such features are completely
encoded in the first few eigenvalues and eigenvectors of the operator
implementing the dynamics. Thus, it all amounts to being able to
approximate such quantities from the high resolution data
and then ensure that the coarse-graining procedure does not perturb
them. 

In this Dissertation, different data-driven techniques addressing various
aspects of molecular coarse-graining will be presented.

First, the problem of distilling a set of physically meaningful collective descriptors from high-resolution data is discussed. 
In particular, a novel strategy (\textit{Variationally optimized Diffusion Maps}) combining existing algorithms to accomplish that is presented, both as validation strategy against different choices of the model parameters, and as an optimized algorithm. Such an approach often requires the computation and storage of large correlation matrices, so a compressed sensing procedure (\textit{oASIS}) is discussed, which allows to fully reconstruct sparse matrices using only a subset of their entries.

Second, the \textit{Structure and State Space Decomposition} ($S^{3}D$) protocol will be discussed, which maps a molecular
primary sequence onto a set of disjoint dynamically coherent domains.
Such units are compelling candidates for effective coarse-grained
degrees of freedom and provide a novel interpretation of the conformational
rearragements the molecule undergoes in terms of splitting and merging
of those units. In particular, results seem to indicate that different
model resolutions may be appropriate for different regions of the
conformational space. 

Next, the \textit{Stepwise Sparse Regressor} and \textit{Spectral Coarse-Graining} will be introduced that
allow to infer the constitutive \textit{renormalized} interactions which
regulate the effective diffusive dynamics of the coarser variables. Both approaches rely on constructing a data-based loss function and optimize its parameters. Preliminary
results on toy-models indicate that both methods consistently capture the long
timescale features expressed by the input data. 

Finally, future developments
and ideas on how to extend the approaches to real molecular systems will be also addressed.


\end{abstract}



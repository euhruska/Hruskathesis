\afterpage{\null\newpage}
\chapter{Adaptive sampling software framework\label{ch:chapter4}}
\section{\label{sec:intro4}Introduction}

One of the motivations of utilizing \emph{Adaptive sampling} is reaching longer timescales in the practical application of sampling biomoleculer and obtaining accurate kinetic behavior. This requires to execute a hierachical ensemble of tasks in a effecient fashion on every level. Without efficient execution of all the subtasks of adaptive sampling the goal of reaching longer timescales wouldn't be realistic. In practice the higher complexity of executing adaptive sampling compare to plain Molecular dynamics reduces the practical utilization of the advantages of adaptive sampling. Here we design and develop a software framework which reduces the entry barrier to implement adaptive sampling by domain experts. Some of the objectives establishing the software frameworks are scalability, maintainability, flexibility and extensibility.
The scalability objective is very important to efficiently utilize the limited computational resources on High-Performance Computers (HPC). The easy maintainability should reduce the entry barrier to execute adaptive sampling. The flexibility and extensibility of this software framework should enable to adjusting for new sampling methods, individual molecular dynamics software choices and different HPCs or supercomputers. In this Chapter we will discuss the development of the \emph{ExTASY} software package and how ExTASY implements the beforementioned objectives.

The material in this chapter was first published following papers: 
\\*
\cite{Extasy2016} Balasubramanian, V.; Bethune, I.; Shkurti, A.; Breitmoser, E.; \textbf{Hruska, E.}; Clementi, C.; Laughton, C.; Jha, S.; Extasy: Scalable and flexible coupling of md simulations
and advanced sampling techniques. Proceedings of the 2016 IEEE 12th
International Conference on e-Science 361{370 (2016).
\\*
\cite{Extasy2019} \textbf{Hruska, E.}; Balasubramanian, V.; Ossyra, J. R.; Jha, S.; Clementi, C.; Extensible
and scalable adaptive sampling on supercomputers. arXiv (2019). URL: https://arxiv.org/abs/1907.06954.


\section{\label{sec:alternative4}Alternative software}

Previously multiple groups have developed software packages which strive to achieve the same objectives for execution of adaptive sampling, but all these framework have some limitations which we attempt to  compared to \emph{ExTASY}.
The package HTMD\cite{doerr2016htmd} has shown the effective adaptive sampling performance for small proteins, including effective retrieval of kinetic information. The performance for larger proteins and the scalability are yet shown. The entry barrier for utilization of HTMD is increased by not being open-sourced. Further, HTMD supports only one single molecular dynamics engines, and a limited number of adaptive sampling strategies.

The recent package DeepDriveMD \cite{leeDeepDriveMDDeepLearningDriven2019} allows to utilize deep learning in adaptive sampling and is able to scalably execute on heterogeneous software and hardware environments, but DeepDriveMD has not yet released or open-sourced the code yet. 

The SSAGES (Software Suite for Advanced General Ensemble Simulations) \cite{SSAGES} is open-source and flexible, not bound to specific molecular dynamics engines, but the effectivity and scalability for large proteins remains to be shown. Multiple software frameworks \cite{jung2019acp, ribeiro2018tjocp, bonati2019pnasu} show the effective sampling of toy systems and small peptides with the ability of utilizing strategies with neural networks, but haven't shown the extensiblity to larger systems or the scalability to perform efficient for larger applications. 

\section{\label{sec:design}ExTASY schematics}
This complexity of utilizing different software frameworks can detract from applying adaptive sampling or researching novel adaptive sampling methods. One method to reduce this complexity is to build the software in a modular approach. For ExTASY the modularity starts from splitting the adaptive sampling into individual steps shown in Figure~\ref{fig:schema2}.

\begin{figure}[h]
  \centering
  \includegraphics[width=0.5\linewidth]{figures2/schema1.pdf}
  \caption{Basic schema of adaptive sampling implemented in ExTASY. The
  number of starting conformations, the MD engines, and analysis strategies or Step 3 stop condition are flexible and can be changed. Identical schema to Figure~\ref{fig:schema}.}
  \label{fig:schema2}
\end{figure}

The individual adaptive sampling steps are summarized as follows:
\begin{itemize}
\item Start: Initialization of the start configurations.  Commonly an unfolded structure of the selected proteins is chosen.
\item Step 1: Simulating an ensemble of MD trajectories. The configurations chosen either by Initialization or Step 4. 
\item Step 2: Analysis of the all previously generated MD trajectories. The analysis varies between adaptive sampling strategies.
\item Step 3: Stop condition. Automatic determination if the objective of adaptive sampling is achieved. When the objective is not achieved, proceed iteratively to Step 4. 
\item Step 4: Selection of restart configurations for the next set of MD trajectories based on the analysis results 
\end{itemize}

The beginning step of any adaptive sampling is the initialization. The start configuration from where the adaptive sampling is launching is chosen. For each replica, which depends on the parallelization of the computational resource, a configuration is generated. The start configuration can be all identical or disparate. For proteins commonly a unfolded state is chosen.
In step 1 the MD simulations for all replicas is performed in a parallelized mode. Here most of the computational resources are utilized and scalability of this step is crucial. Commonly  
Following the MD simulation is the analysis of the MD trajectories in step 2. In ExTASY the flexibility python scripts allow the simple implementation and modification of different analysis strategies, including deep learning approaches\cite{Mardt2018}. In the following Chapters 5 and 6 the adaptive sampling strategies $cmicro$ and $cmacro$ with Markov State Models \cite{prinz2011markov} are used as described in Chapter 3. 

The results of Step 2 allow to determine in Step 3 if the objectives of adaptive sampling are reached. Examples for the objectives are protein folding, achieving a certain accuracy in protein dynamics or exploration of a part of the energy landscape corresponding to a smaller-scale motion of the protein. If the objective is not reached, Step 4 creates the restartings configuration for the next MD trajectories corresponding to Step 1. The adaptive sampling strategies to determine the restarting points in Step 4 in Figure~\ref{fig:schema2} are easily exchangeable, such as the strategies discussed in Chapter 2 or the FAST method \cite{FAST}. Once the objective in Step 3 is reached, adaptive sampling finishes and all trajectories can be further analyzed. 

\section{\label{sec:asynch}Asynchronous execution}

One crucial feature of ExTASY is the ability for asynchronous execution of the individual subtasks. As far we know, no other adaptive sampling platform enables asynchronous execution. This significantly improves the efficiency of adaptive sampling in practice. In the standard, synchronous execution, all previous steps have to be conclude to begin the next step. 
The aforementioned Steps 2-4 require very low parallelization compared to Step 1. In a typical use case Step 1 utilizes a large number of GPUs due to the large number of replicas simultaneously simulated. Once Step 1 finishes then Step 2-4 are run, which typically utilize a fraction of nodes or GPUs compared to Step 1. When running on a constant number of nodes or GPUs, which is the most common case on HPCs, this low utilization of nodes for Step 2-4 lead to reduced. This performance penalty is larger when the adaptive sampling steps take a longer time to finish, for example for a more complex strategy or for larger biomolecules. The asynchronous execution shown in Figure~\ref{fig:asynch} allows to increase the utilization of computational resources and consequently reaching longer timescales by continuously utilizing all nodes or GPUs and removing the downtimes for MD workers. The MD tasks require new restarting configurations to start the simulations. These new restarting configurations are continuously updated by repeatedly running the adaptive sampling analysis steps. This ensures the restarting configuration which take into account the recently partially unfinished molecular dynamics tractories. Due to the time delay of finishing adaptive sampling analysis a small fraction at the end of MD trajectories won't be analysed, but in the next restarting point these small parts will be analysed.
The first version of ExTASY\cite{Extasy2016} had only synchronous execution enabled, but the current \cite{Extasy2019} includes asynchrounous execution. In chapter 5 and 6 all the adaptive sampling is executed asynchronously. 

\begin{figure}[h]
  \centering
  \includegraphics[width=0.6\textwidth]{figures2/asynch.pdf}
  \caption{Asynchronous execution of the ensemble of adaptive sampling subtasks, including molecular dynamics and analysis tasks. The restart configurations for the next molecular dynamics simulation are determined by the last finished analysis task. Here the steps 2 and 4 are shown combined as analysis steps.}
  \label{fig:asynch}
\end{figure}



\section{\label{sec:Tools}Tools and Software}

ExTASY was built by utilizing the RADICAL-Cybertools \cite{Balasubramanian2019rct}. The RADICAL-Cybertools is a software systems designed to execute ensembles of tasks on HPC platforms in a scalable and modular approach.

ExTASY uses the RADICAL-Cybertools component EnTK (Ensemble Toolkit)
\cite{entk-icpp-2016, balasubramanian2018harnessing} to communicate with RADICAL-Cybertools. The EnTK layer enables to abstract the execution of individual tasks from the explicit resource
management. In the background, hidden from ExTASY, RADICAL-Cybertools performs the explicit resource acquisition and resource management in a scalable fashion as well error tracking and analytics. The whole RADICAL-Cybertools was built in a blocks approach \cite{turilli2018building}, allowing modularity and flexibility for ExTASY.


\begin{figure}[h!]
    \begin{lstlisting}
    from @radical.entk@ import @Task, Stage, Pipeline@
    p = Pipeline()

    sim_stage = Stage()   
    sim_task = Task()
    sim_task.executable = @<executable>@ #example openmm
    sim_task.arguments = <args> #example openmm args
    <add other task properties>
    sim_stage.add_tasks(sim_task)
    
    ana_stage = Stage()
    ana_task = Task()
    ana_task.executable = @<executable>@ #example pyemma
    ana_task.arguments = <args> #example pyemma args
    <add other task properties>
    
    ana_stage.add_tasks(ana_task)
    ana_stage.post_exec = {
        'condition': eval_sims(),
        'on_true':   add_sims(),
        'on_false':  terminate()
        }
    
    p.add_stages([sim_stage, ana_stage])
    \end{lstlisting}
    \caption{Pseudocode showing the modular software design and the EnTK API backend}\label{extasy_snippet}
\end{figure}


Figure~\ref{extasy_snippet} shows pseudo-code illustrating the ExTASY implementation of adaptive sampling with EnTK API. The individual separtion of simulation and analysis tasks is recognizable. Figure~\ref{fig:extasy_arch} shows the modular integration between ExTASY and EnTK. ExTASY describes the adaptive sampling as an set of executable tasks. ExTASY determines the parameters for the Resource description (event 1) and the MD and analysis tasks (event 2). ExTASY then passes these parameters to the EnTK's interface which defines the resource and application (event 3 and 4). Once defined, EnTK starts execution of the executable ExTASY tasks on the target resource (event 5, 6 and 7). 

\begin{figure}[H]
 \centering
  \includegraphics[width=0.8\textwidth]{figures2/extasy_arch_coarse.pdf}
  \caption{EnTK backend: This schema shows the modularity which enables ExTASY's scalability which ensuring flexibity.  
  }\label{fig:extasy_arch}
\end{figure}


\section{Pilot abstraction} 
One reason for ExTASY to utilize RADICAL-Cybertools is the improved ability of executing tasks on computational resources. Traditionally multiple HPC tasks can be execute either as individual jobs or with message-passing interface (MPI) as part of a single job. The first approach is effective for independent tasks, but for interdependent tasks as in adaptive sampling this approach is suboptimal. The second method is suboptimal for the heterogeneous and adaptive tasks in adaptive sampling. The pilot abstraction in RADICAL-Cybertools~\cite{turilli2018comprehensive} overcomes these limitations.  The pilot abstraction separates the initial computational resource acquisition from individual task-to-resource assignments. First, the computational resources are requested with placeholder jobs without any tasks. In the second step, while the job is running, individual tasks are asigned to the placeholder jobs. RADICAL-Cybertools is engineered to support scalable pilot abstraction for launching heterogeneous tasks across different platforms. This pilot abstraction of the RADICAL-Cybertools allows ExTASY to effectively execute all the individual subtasks of adaptive sampling.  

\section{\label{sec:scaling}Scaling}


One of the main metrics to deploy tools to Supercomputers is the scalability. Only algorithms with sufficient parallelization can utilize these computational resources effectively. The scalability of adaptive sampling depends both on the scalability of the software platform and the scalability of the individual MD engines and analysis tools. To abstract only the scalability of the software platform we define the efficiency as the ratio between the nodehours used by all the individual tasks and the nodehours used by the software platform. 
This efficiency was measured for increasing parallelizations up to 2000 GPUs on Summit, shown in Figure~\ref{fig:scaling}. With the asynchronous execution one GPU was tasked to run analysis tasks, all other GPUs were running MD tasks. 6 GPUs per node and 2 hour long computational jobs were utilized. The efficiency of ExTASY above 2000 GPUs is reduced due to the time delay caused by communication between individual RADICAL components. RADICAL-Cybertools adapts to the specific software environments of the HPCs\cite{turilli2019ac}, improving the scalability and illustrating the advantage of platform agnostic execution and portability across
heterogeneous HPC resources. The asynchronous version of ExTASY allows a scaling to a higher number of nodes than alternative adaptive sampling software platforms.

\begin{figure}[H]
  \centering
  \includegraphics[width=0.7\textwidth]{figures2/plot_scalingefficiencylog.pdf}
  \caption{Scaling of ExTASY on Summit. The Efficiency is defined as the ratio between the nodehours used by all the individual tasks and the nodehours used by the software platform. }
  \label{fig:scaling}
\end{figure}



\section{\label{sec:conclusion}Conclusion}
This thesis shows the development of the ExTASY framework, which achieves several objectives to enable effective execution of adaptive sampling workloads. A scalability of over 1000 GPUs and 1000 replicas was demonstrated on the supercomputer Summit. This scalability reduces the technical barrier to adaptive sampling for larger proteins. Notably, this scalability doesn't reduce the flexibility of the platform.

The open-source nature of the The ExTASY framework is an important feature, the code is available at \href{https://github.com/ClementiGroup/ExTASY}{https://github.com/ClementiGroup/ExTASY}. Bu releasing the platform open-source the implementation of new adaptive sampling strategies is simplified. This flexibility allowed ExTASY to be the first open-source, adaptive sampling platform which supports deep learning or asynchronous execution. The asynchronous execution significantly improves the utilization of computational resources for larger proteins or more complex adaptive sampling strategies.

The modularity of this software plaform enables multiple objectives. One objective is the agnostism from the HPC platform, preventing code working only for specific software and hardware environments. The modularity also improves the maintainability, reliability and reproducibility of adaptive sampling. 
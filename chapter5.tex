\afterpage{\null\newpage}
\chapter{Aplication of adaptive sampling\label{ch:chapter5}}

The development of an effective, scalable software platform in Chapter 5 allows us to execute adaptive sampling for larger application and larger proteins.
In this chapter adaptive sampling will be used to determine the folding and the conformational dynamics of 4 proteins. These results allows us to compare the effeciency and accuracy of adaptive sampling compared with plain molecular dynamics.  These 4 proteins range from 10 residues to 73 residues. The computational resources available limited the application to these small, fast folding proteins. 

The material in this chapter was first published in following paper: 
\\*
\cite{Extasy2019} \textbf{Hruska, E.}; Balasubramanian, V.; Ossyra, J. R.; Jha, S.; Clementi, C.; Extensible
and scalable adaptive sampling on supercomputers. arXiv (2019). URL: https://arxiv.org/abs/1907.06954.


\section{\label{sec:intro5}Introduction}
  
In Chapter 4 the efficiency and reliability of adaptive sampling strategies was compared in a statistically significant approach.  In that chapter Markov Chain trajectories were utilized to approximate the behavior of molecular dynamics trajectories. Since Markov Chain trajectories can be generated much faster than molecular dynamics trajectories a statistically significant reesults could be achieved. The limitation of this approach is that Markov Chain trajectories are only approximations and full molecular dynamics trajectories are necessary to confirm the performance of adaptive sampling. From Chapter 4 we expect a higher speed up with adaptive sampling\cite{Adstrategies2018} compared to plain MD for larger proteins.

\section{\label{sec:Reference}Reference Data}



To compare the speed up and accuracy of the adaptive sampling methods compared to plain MD we utilize reference proteins and reference trajectories to compare the results.
The 4 small proteins have long reference simulations generated on the Anton supercomputer \cite{lindorff2011}, which fold and unfold each of the proteins multiple times. The multiple folding events ensure a well sampled energy landscape and a good accuracy for the reference comparison. To fold and unfold proteins multiple times requires large computational resources, laving only for very limited number of proteins as well sampled reference data with a well sampled energy landscapes and conformational dynamics.

\begin{table}[H]
\centering
%\begin{tabular}{ccccc}
\resizebox{\columnwidth}{!}{
\begin{tabular}{|c|c|c|c|c|c|}
\hline
Protein & PDB ID & \# of Residues & Folding Time ($\mu$s) \cite{lindorff2011} & Unfolding Time ($\mu$s)\\ 
\hline
Chignolin    & 5AWL                       & 10                 & 0.6                &2.2            \\
Villin       & 2F4K                       & 35                 & 2.8                &0.9            \\
BBA          & 1FME                       & 28                 & 18                 &5              \\
A3D          & 2A3D                       & 73                 & 27                 &31              \\
\hline            
\end{tabular}
}
\caption{Proteins for the comparison of adaptive sampling and plain MD. }\label{table:dataset-summary}
\end{table}
The folding and unfolding times of the 4 proteins are shown in Table \ref{tab:dataset-summary}. All the folding times are below 40 $\mu$s, a timescale reachable with the available resources. To compare the adaptive sampling and plain MD we ran for all proteins both methods with ExTASY on the Summit supercomputer. To ensure comparability, the parallelization was the same for adaptive sampling and plain MD.


\section{\label{sec:MD}Setup}

The initialization of both plain MD and adaptive sampling was identical. For the 4 proteins a single start frame was chosen from reference dataset. To ensure this frame was unfolded, random selection was only among the 20\% frames with the highest Root Mean Square Deviation to the
protein crystal structure. The initial coordinates for ExTASY we obtained after a short energy equilibration (1-2ns) in an NPT ensemble. 

The parameters for the MD simulations were following. The setup of the reference trajectories\cite{lindorff2011} was replicated, with a CHARMM22* force field \cite{Charmm22star}
and the modified TIP3P water. Table \ref{table:param1} shows the identical temperaturs of the MD simulation to the reference simulations. One difference was the use of the Particle Mesh Ewald method for long-range electrostatics in OpenMM. The MD stepsize was 5fs and 2fs for A3D. The MD was run on OpenMM 7.5 \cite{openMM} using CUDA 9.1.  

For both adaptive sampling and plain MD 50 replicas were used, each replica simulated on one GPU. The MD trajectory lengths were 50ns for Chignolin and Villin, 10ns for BBA and 40 ns for A3D.  The resulting trajectories were strided to reduce the data volume. Two copies of each trajectory were generated. One trajectory stored all information including water and one trajectory stored only the protein coordinates. The protein-only trajectories were used to speed up the analysis, which was limited by the data transfer rate.


The analysis for adaptive sampling was run on one GPU in an asynchronous manner as shown in Chapter 5. The parameters for the analysis step of the adaptive sampling strategies were the following.  For dimension reduction deep learning with SRVs was utilized, as introduced in Chapter 2.
The SRV features were the distances between all pairs of C-alpha atoms. For protein A3D only th distance pairs between every second C-alpha atom were selected due to the larger size of the protein.  For proteins Chignolin, Villin and BBA the SRVs dimension reduction was run with 8 deep learning epochs, 6 hidden layers, size of hidden layers 50, activation tanh, learning rate 1e-4, dropout rate 0.1 and latent space noise 0.1. For the larger protein A3D, the SRV was speed up with following parameter: 1 deep learning epoch, 2 hidden layers, size of hidden layers 160, activation function elu (Exponential linear unit), learning rate 1e-3, dropout rate 0.01 and latent space noise 0.1. The number of SRV output dimensions was 10 for Chignolin and 4 for Villin, BBA and A3D. 

The SRV lag times and MSM lag times are shown in Table \ref{table:param1}. For BBA short lag times wer chosen to investigate the effect of shorter trajectory lengths. These shorter trajectory length were enabled by using SRVs instead of TICA. SRVs reach identical accuracy with shorter time lags than TICA dimension reduction\cite{chen2019jcp}. The number k-means clustering states fro the MSMs was 500 and 200 for protein A3D. For Chignolin and A3D the chosen adaptive sampling strategy was $cmacro$, for Villin and BBA $cmicro$ adaptive sampling strategy was used. For the $cmacro$ strategy 20 macrostates were used.

Once both the adaptive sampling and plain MD was concluded, the exploration and the protein dynamics were compared with the help of the Anton MD reference trajectories. For the final analysis, TICA dimension reduction with kinetic map scaling and Koopman correction was used with a lag time of 10ns or 1.5ns for BBA. The TICA input features were the distances and inverse distances between all pairs of C-alpha atoms. After projection of both the adaptive sampling and plain MD trajectories on the reference trajectories in the TICA space, the exploration and protein dynamics were compared.

\begin{table}[H]

\centering
%\begin{tabular}{ccccc}
\resizebox{\columnwidth}{!}{
\begin{tabular}{|c|c|c|c|c|c|c|c|}
\hline
Protein Name  & MD Temperature [K]  &MD [ns]/iteration  & MD step size [fs] & SRV lag [ns] & MSM lag [ns]    \\ 
\hline
Chignolin      & 340 &   50 & 5 &1.5 & 10     \\
Villin         &   360 & 50 & 5 & 1.5 & 10         \\
BBA            &   325 &10 & 5 & 1.5 & 1      \\
A3D            &  370 & 40 & 2 & 10 & 10    \\
\hline            
\end{tabular}
}
\caption{ExTASY parameters for MD and analysis, for both plain MD and adaptive sampling.  }\label{table:param1}
\end{table}


\section{\label{sec:results}Results}

The three main considerations when comparing adaptive sampling and plain MD were the speed up achieved, exploration of the whole energy landscape and the accuracy of the conformational dynamics.
The exploration of the whole energy landscape was shown by projecting the adaptive sampling onto the reference trajectories in the TICA space. By measuring the explored fraction of the population of simulation time the speed up compared to plain MD could be determined. The accuracy of the protein dynamics was determined by analyzing the entropy of the MSM transition
matrices and the Mean First Passage Time (MFPT).



\subsection{\label{sec:time-fold}Comparison of Exploration}

The high-dimensional nature of the whole energy landscape of biomolecules prevents to visualize and compare the whole energy landscape. Only the dimension reduced free energy landscape of the proteins in TICA coordinates was compared, shown in Figure~\ref{fig:overlap}. The background in color shows the reference free energy landscape of the proteins and the shaded foreground is the area which adaptive sampling successfully explored. For all four proteins, Chignolin, Villin, BBA and A3D, the whole energy landscapes was explored by adaptive sampling, including the folded state. This shows that adaptive sampling not ony folded these proteins, but correctly recovered the energy landscape for these proteins.  The small differences compared to the reference energy landscapes predominantly occour in the rare sampled areas of the energy landscapes. This discrepancy can be explained by the stochasticity of both the adaptive sampling and reference MD simulation. Despide the relatively long data size the stochasticity causes only poor sampling on the energy landscape in some low probability areas of the energy landscapes.  


\begin{figure}[H]
\centering
   \begin{subfigure}[b]{0.6\linewidth}
   {\scalebarimg{figures2/plot_reana_61_extasy_vamp_chignolin3-free-energy-TICA2-overlap_mod.pdf}{0.98}{A)}{0mm}{-2mm}}
   \end{subfigure}%
   
   \begin{subfigure}[b]{0.6\linewidth}
   {\scalebarimg{figures2/plot_reana_61_extasy_vamp_villin3-free-energy-TICA2-overlap_mod.pdf}{0.98}{B)}{0mm}{-2mm}}
   \end{subfigure}%


  \caption{The recovery of the energy landscape by adaptive sampling for 4 proteins. The projection on TICA coordinates is shown. The
 color background shows the reference Free Energy landscape and the black diagonal lines show the adaptive sampling explored energy landscape. The location of the folded states are shown.  A) Chignolin B) Villin }
\end{figure}

\begin{figure}[H]\ContinuedFloat
\centering
   \begin{subfigure}[b]{0.6\linewidth}
   {\scalebarimg{figures2/plot_reana_61_extasy_vamp_bba3-cmicro-free-energy-TICA2-overlap_mod.pdf}{0.98}{C)}{0mm}{-2mm}}
   \end{subfigure}%
  
   \begin{subfigure}[b]{0.6\linewidth}
   {\scalebarimg{figures2/plot_reana_61_extasy_a3d9-free-energy-TICA2-overlap_mod.pdf}{0.98}{D)}{0mm}{-2mm}}
   \end{subfigure}%

  \caption{(cont.) C) BBA D) A3D} 
  \label{fig:overlap}
\end{figure}

To compare the speed of exploration between adaptive sampling and plain MD, the fraction of explored population over time can be compared, as shown in Figure~\ref{fig:Pop_explored}. The time here is measure in absolute simulation time. The absolute simulation time is the length of consecutive molecular dynamics simulations, omitting any parallel similations. For example the absolute simulation time is the length of one single trajectory multiplied the number of iterations. This metric is directly related to the time to solution, which is in most cases the relevant criterium if an adaptive sampling can be performed. The absolute simulation length is hardware independent, which simplifies the comparison, in contrast to time to solution which heavily depends on hardware and software choices. Chapter 4 explored the effect of parallelization on the absolute simulation time, in this chapter the parallelization is held constant.

The fraction of explored population is measured as a function of increasing absolute simulation time during an exploration, either for adaptive sampling or for plain MD. The populations and states for the the calculation of the explored population are defined by the reference datasets. The fraction of explored population is calculated by deteemining which states were explored and weighting with the stationary weight from the reference dataset.

ExTASY can flexibly implement different adaptive sampling strategies. To illustrate the flexibility of ExTASY the proteins Chignolin and A3D were adaptively sampled with the $cmacro$ strategy and protein Villin and BBA were adaptively sampled with the $cmicro$ strategy. The strategies are described in Chapter 3. 

Figure~\ref{fig:Pop_explored} shows that adaptive sampling both explores the landscape faster and also folds the 4 proteins faster than plain MD. The speed up of folding with adaptive sampling is 170\% for Chignolin, 20\% for Villin, 380\% for BBA, and above 690\% for A3D. For A3D this value couldn't be accurately determined due to limited computational resources to fold A3D with plain MD. A3D shows that even while the speed up with adaptive sampling increases for larger proteins, the computational resources necessary to fold the larger proteins increase too. The increased speed up with adaptive sampling for A3D follows the predictions discussed in Chapter 4.  

\begin{figure}[H]
\centering
   \begin{subfigure}[b]{0.6\linewidth}
   {\scalebarimg{figures2/plot_chigsummary14_pop_fraction2_2_mod3.pdf}{0.95}{A)}{0mm}{-5mm}}
   \end{subfigure}%
   
   \begin{subfigure}[b]{0.6\linewidth}
   {\scalebarimg{figures2/plot_vilsummary14_pop_fraction2_2_mod3.pdf}{0.95}{B)}{0mm}{-5mm}}
   \end{subfigure}%


  \caption{The fraction of explored population increases with absolute simulation time. The absolute simulation is shown with an logarithmic scale. Folding events are pinpointed by the vertical lines. The speed up with adaptive sampling depends on the protein and generaly increases with the size of the protein. A) Chignolin B) Villin }
\end{figure}

\begin{figure}[H]\ContinuedFloat
\centering
   \begin{subfigure}[b]{0.6\linewidth}
   {\scalebarimg{figures2/plot_bbasummary14_pop_fraction2_2_mod3.pdf}{0.95}{C)}{0mm}{-5mm}}
    \end{subfigure}%

   \begin{subfigure}[b]{0.6\linewidth}
   {\scalebarimg{figures2/plot_a3dsummary15_pop_fraction2_2_mod3.pdf}{0.95}{D)}{0mm}{-5mm}}
    \end{subfigure}%
  \caption{(cont.)  C) BBA D) A3D For protein A3D the limited computational resources interrupted the folding with plain MD. The adaptive sampling of A3D was able to fold due to the speed up of adaptive sampling.}
  \label{fig:Pop_explored}
\end{figure}





\subsection{\label{sec:kinetics}Comparison of Protein Dynamics}



A popular and robust metric to measure the kinetic behavior of proteins in the Mean First Passage Time (MFPT). The MFPT measures the mean time a trajectory takes to reach state B for the first time while starting from state A. In this chapter MFPT from the folded states to the unfolded states is measured. This metric is a relatively robust measure of the conformational dynamics. A low error in the measured mfpt can confirm that the conformational dynamics are sampled accurately. The ensemble of folded and unfolded states are defined for each protein by the TICA coordinates of the states. 

Figure~\ref{fig:mfpt} shows the evolution of mfpt with the absolute simulation time for individual proteins. At the beginning of the simulations, when the folded state is not explored yet, the mfpt can not be calculted. The MFPT for both plain MD and adaptive sampling converges towards the reference dataset value. The remaining errors of simulated MFPT compared to the reference MFPT are about one order of magnitude for both adaptive sampling and plain MD. When comparing the error of adaptive sampling MFPT with plain MD, the protein BBA shows as smaller adaptive sampling error, Villin shows similar error and Chignolin shows larger adaptive sampling error. The interpretation of the relative size of the MFPT errors is limited due to the small sample size and large stochasticity inherent to molecular dynamics. This allows us to conclude that adaptive sampling reaches similar errors of configurational dynamics in a much short time due to the speed up. Novel adaptive sampling strategies could reach higher accuracies than plain MD. 


These magnitude of MFPT errors matches the results obtained with the HTMD framework \cite{doerr2016htmd} for the protein Villin. With ExTASY the conformational dynamics of larger proteins such as BBA was investigated. The results for A3D are limited due to the limited computational resources preventing the folding of A3D with plain MD.  
\begin{figure}[H]
\centering
   \begin{subfigure}[b]{0.5\linewidth}
   {\scalebarimg{figures2/plot_chigsummary14_mfpt_unfold_3_mod2.pdf}{1}{A)}{0mm}{-2mm}}
   \end{subfigure}%
   
   \begin{subfigure}[b]{0.5\linewidth}
   {\scalebarimg{figures2/plot_vilsummary14_mfpt_unfold_3_mod2.pdf}{1}{B)}{0mm}{-2mm}}
   \end{subfigure}%
   
   \begin{subfigure}[b]{0.5\linewidth}
   {\scalebarimg{figures2/plot_bbasummary14_mfpt_unfold_3_mod2.pdf}{1}{C)}{0mm}{-2mm}}
    \end{subfigure}%
  \caption{
  The evolution of the unfolding Mean First Passage Times is shown over the simulation time. Adaptive sampling is red and plain MD is blue. The reference values are shown in black.
  A) Chignolin B) Villin C) BBA } 
  \label{fig:mfpt}
\end{figure}


Another metric to inspect the accuracy of the configurational dynamics is relative entropy \cite{bowman2010enhanced}. This modified entropy compares the ransition matrices between reference and the sampled data. One motivation of using relative entropy that this measure includes all transitions, as opposed to MFPT which focuses on certain configurational motions. 
The relative entropy compares the MSM transition matrices of the reference data $P_{ij}$ and of the analyzed data $Q_{ij}$. The two transition matrices are required to use exactly the same dimension reduction and clustering for generation. The relative entropy $D(P||Q)$ is defined as following where $s_{i}$ is the equilibrium probability of state $i$. This equation is analogous to the Kullback–Leibler divergence equation.

\begin{equation}
D(P||Q)=\sum_{i,j}^{N}s_{i}P_{ij}\ln\frac{P_{ij}}{Q_{ij}}. 
\end{equation}

To make the relative entropy metric more robust to the stochasticity of the sampling, a correction of the transition matrices is added. A pseudo-count of $1/N$ is added to each element of the count matrices, with $N$ the number of states. The transition matrices are obtained by before normalizing the rows of the count matrices. This pseudo-counts represents a uniform prior and reduces the effect of zero values in $Q_{ij}$.

Figure ~\ref{fig:rel_ent} shows how evolution of the relative entropy with absolute simulation time. Increasing sampling for both plain MD and adaptive sampling decreases the relative entropy in direction of zero. It;s observed that adaptive sampling for Villin reduces faster the relative entropy at the beginning than plain MD. In later stages of simulation, the plain MD reduces the relative entropy faster. The explanation for the later slower reduction of relative entropy is that the adaptive sampling strategies prioritize the sampling of slow collective motions. In contrast the fast transitions are sampled less. The relative entropy depends mostly on the fast transitions due to the larger number of fast transitions. Relative entropy should be used only for use cases where the sampling accuracy of many fast transitions is the objective.

\begin{figure}[H]
   \centering
   \begin{subfigure}[b]{0.7\linewidth}
   {\scalebarimg{figures2/plot_vilsummary14_rel_ent_fold_4_mod2.pdf}{1}{}{0mm}{-3mm}}
   \end{subfigure}%
   
  \caption{
  The evolution of the relative entropy between the MSM transition matrices from adaptive sampling and plain MD is shown.  For protein Villin the relative entropy decreases with increasing simulation time.}
  \label{fig:rel_ent}
\end{figure}


\section{\label{sec:conclusion}Conclusion}

The results in this Chapter show the ExTASY framework \cite{Extasy2019} effectively
performed adaptive sampling, confirming that adaptive sampling performs as predicted for the investigated 4 proteins. 
For all proteins the free energy landscape was fully sampled, up to the expected stochasticity on rarely sampled areas. The speed up of folding with adaptive sampling reached values between 20\% to 690\%, where the largest increase was observed for the largest protein. This observation falls in line the prediction that larger, and more complex protein reach larger speed ups with adaptive sampling. the values of speed up fall also in the expected range. Due to the large stochasticity and low sample size caused by limited computational resources, these exact values of speed up have limited statistical significance. 

One of the claimed advantages of adaptive sampling compared to other sampling techniques is the accurate estimation of kinetic values. The observed Mean First Passage Times confirm that adaptive sampling reaches similar accuracy of kinetic values after a short absolute simulation length. The absolute simulation length is directly related to the time to solution. The observed errors of kinetic values for both adaptive sampling and plain MD are up to a order of magnitude and further approaches to reduce the size of the errors are necessary. Novel adaptive sampling strategies optimized specifically for accuracy of slow conformational dynamics would improve the results of adaptive sampling compared to plain MD. The comparison of relative entropy can conclude that relative entropy has only limited informative value for the accuracy of the slowest motions of the biomolecules.  
 

Due to the results matching with the results in Chapter 4, we can conclude that the approach of approximating molecular dynamics trajectories with Markov Chain trajectories doesn't introduce significant errors. The resulting predictions in adaptive sampling performance were confirmed with molecular dynamics trajectories. The large stochastic component in the performance of adaptive sampling was observed as well.


The limited computational resources only allowed to confirm the performance for proteins up to 73 residues. Larger protein with longer folding times would require longer MD simulations, but the speed up is expected to increase. For larger protein even larger number of replicas than 50 would be effective, as shown in Chapter 4. The scaling of adaptive sampling performance to larger number of replicas was predicted to increase for larger proteins. No software scaling limitating are expected for larger proteins, since the software package ExTASY scales up to 1000s of GPUs which represents up to 1000 replicas.
\documentclass[12pt]{ruthesis} 
\usepackage{amsmath}
\usepackage{amssymb}
\usepackage{latexsym}
\usepackage{graphics}
\usepackage{epsfig,epsf,rotating}
\usepackage{subcaption}
%\usepackage{pictex}
\usepackage{epsf}
\usepackage{cite}
\usepackage{color}
\usepackage{xcolor}
\usepackage{theorem}
\usepackage{hyperref}
\usepackage{xspace}
\usepackage{float}
\usepackage{listings}
\usepackage{graphicx}
\usepackage[utf8]{inputenc}
\usepackage{textalpha}
\usepackage{tikz}
\usepackage{listings}
\usepackage{afterpage}
\setcounter{tocdepth}{4}
\setcounter{secnumdepth}{4}
\DeclareUnicodeCharacter{0301}{ }
\newtheorem{proposition}{Proposition}

\theoremheaderfont{\itshape} {\theoremstyle{break}
\newtheorem{Fact}{Fact}[chapter]} \theoremstyle{break}
\newtheorem{Lem}{Lemma}[chapter] \theoremstyle{break}
\newtheorem{Thm}{Theorem}[chapter] {\theoremstyle{plain}
  \theorembodyfont{\rmfamily}  \newtheorem{Prf}{Proof}[chapter]}
{\theoremstyle{plain}
  \theorembodyfont{\rmfamily}  \newtheorem{Def}{Definition}[chapter]}

\newcommand\GRE[1]{\textcolor{green}{\textbf{#1}}}
\newcommand\RED[1]{\textcolor{red}{\textbf{#1}}}

\newcommand{\scalebarimg}[5]{
  \centering
  \begin{tikzpicture}
    \draw node[name=micrograph] {\includegraphics[width=#2\linewidth]{#1}}; %I fetch the image
    \draw  (micrograph.north west)  node[anchor=north west,yshift=-#4, xshift=#5,black]{\textbf{\large{#3}}}; %I draw the image label
  \end{tikzpicture}
}



\definecolor{background}{rgb}{0.94,0.95,0.96}
\lstset{ 
  % backgroundcolor=\color{white},
 basicstyle=\ttfamily\scriptsize, % the size of the fonts that are used for the code 
 breakatwhitespace=false, % sets if automatic breaks should only happen at whitespace 
 breaklines=true, % sets automatic line breaking 
 backgroundcolor=\color{background},
 captionpos=b, % sets the caption-position to bottom 
 commentstyle=\color{gray}, % comment style 
 escapeinside={\%*}{*)}, % if you want to add LaTeX within your code 
 extendedchars=true, % lets you use non-ASCII characters; for 8-bits encreadreading twitter ing twitter odings only, does not work with UTF-8
 frame=single, % adds a frame around the code 
 keepspaces=true, % keeps spaces in text, useful for keeping indentation of code (possibly needs columns=flexible)  
 keywordstyle=\color{blue}, % keyword style 
 language=Python, % the language of the code 
 numbers=none, % where to put the line-numbers; possible values are (none, left, right) 
 numbersep=5pt, % how far the line-numbers are from the code 
 numberstyle=\tiny\color{gray}, % the style that is used for the line-numbers 
%  rulecolor=\color{white}, % if not set, the frame-color may be changed on line-breaks within not-black text (e.g. comments (green here)) 
 showspaces=false, % show spaces everywhere adding particular underscores; it overrides 'showstringspaces' 
 showstringspaces=false, % underline spaces within strings only 
 showtabs=false, % show tabs within strings adding particular underscores 
 stepnumber=2, % the step between two line-numbers. If it's 1, each line will be numbered 
 stringstyle=\color{green}, % string literal style 
 tabsize=2, % sets default tabsize to 2 spaces 
 moredelim=**[is][\color{red}]{@}{@},
 xleftmargin=1em,
 framexleftmargin=-1.5em,
}

\title{Adaptive sampling of Conformational Dynamics}
\ctitle{Adaptive sampling of Conformational Dynamics}
\author{Eugen Hruška}
%\department{Electrical and Computer Engineering}
\school{Rice University}
\degree{Doctor of Philosophy}

\committee {
        Cecilia Clementi, Thesis Director\\
Professor of Chemistry and Chemical \&
Biomolecular Engineering\and
Jose Onuchic, Chair \\
        Harry C. and Olga K. Wiess Chair of
Physics, Professor of Chemistry and
BioSciences \and
        Jason Hafner \\
        Professor of Physics \& Astronomy and
Chemistry \and
        Matteo Pasquali\\
        A. J. Hartsook Professor of Chemical and
Biomolecular Engineering, Chemistry,
Material Science and NanoEngineering 
}

\address{Houston, Texas}
\donemonth{April} \doneyear{2020} \makeindex
\begin{document}

  \begin{frontmatter}
   \pagenumbering{roman}
   %\makecover
   \maketitle
   
   %\include{ded}
   \afterpage{\null\newpage}
\thispagestyle{empty}
\begin{abstract}

The challenge of predicting the conformation dynamics of biomolecules is at the core of our limited ability to understand many biophysical processes. This includes many questions around the causes of many diseases and biophysics theory. \emph{Adaptive sampling} is an approach to increase our ability to predict conformational dynamics. 

The challenge of effectively sampling the conformation dynamics of biomolecules is one example of a more general problem.  In terms of physics, the general problem is the accurate sampling of the time-dynamics of high-dimensional stochastic systems. The high-dimensionality combined with a complex energy landscape prevents any simple solutions. 

Many approaches to unravel this challenge have in the past achieved significant improvements. In the case of proteins, the timescales where we are able to predict the conformational dynamics increased by many orders of magnitudes to the millisecond scale. This illustrates the magnitude of the challenge since the current state-of-art can only predict the accurate behavior for small proteins. For most of the larger biomolecules, we are not able to predict the accurate behavior. This is not only caused by the several magnitude longer timescales but also an order of magnitude larger sizes of the biomolecules.  

Considering the broad scope of the general challenge for sampling a high-dimensional stochastic system, this Dissertation will focus only on improving the prediction of conformational dynamics of proteins. This restriction doesn't limit the applicability of the methods investigated to other high-dimensional stochastic systems.

First, the prediction of the effectivity of different adaptive sampling strategies will be discussed. Due to significant stochasticity and protein-to-protein variation, the choice of adaptive sampling strategy is not apparent. The performance of different goals varies as well. 

Second, to deepen our theoretical understanding of adaptive sampling strategies, an upper limit for the performance of any adaptive sampling strategy is developed. This theoretical upper limit allows us to understand the potential and limits of adaptive sampling.

Third, adaptive sampling is heavily dependent on software due to the thousands or millions of individual steps necessary to be performed, all to be executed in an efficient fashion on a High-Performance Computer (HPC). Here we show the development of the software package \emph{ExTASY}. This framework allows executing all the necessary steps for adaptive sampling while reducing the work-load. The innovations of ExTASY are both the high-scalability and the modularity, which allow for an easy change of the adaptive sampling strategies and maintenance. 

Finally, the package \emph{ExTASY} will be applied to show the results of adaptive sampling for several proteins. Future developments to extend the investigated approaches to longer timescales will be addressed.

\end{abstract}



   % star needed to make sure that the 'acknowledgments' are not included in chapter numbering
\chapter*{Acknowledgments}


This dissertation is just the final product of a long path, starting when I joined Rice University in 2014.
Along this long path, I have received lots of help and lots of advice. Here I want to thank everyone even though 
I can only mention some.

At the center of this path is Cecilia Clementi. As my advisor, she guided me through the ups and downs, helping me understand why something did and did not work. This has helped me grow professionally
and personally.
She consistently encouraged me to improve my results and persist when software bugs caused bad results. Without her, this work would not have been possible; she has been an incredible advisor. 

All my previous and current lab mates played a significant role in helping me along this path. Jordane and Lorenzo helped me from the start to learn the crucial as well the less obvious skills for this path. Many thanks to Shantenu, Vivek, and Jayvee for being great collaborators and for being always willing to help, whatever the question.

I would also like to thank Jose Onuchic, Jason Hafner, and Matteo Pasquali for agreeing to sit on my PhD defense committee.

Last but not least, I want to thank my family for the unconditional support. Thank you.

   %\include{ack}
   \tableofcontents
   \listoffigures
   \listoftables
%   \include{ded}
  \end{frontmatter}
\pagenumbering{arabic}

\linespacing{1.7}

%\chapter{Motivation and Preface}
\label{ch:Intro}
\GRE{This thesis is about adaptive sampling of conformational dynamics, a method to efficiently obtain relevant information about the stationary and kinetic behavior of high-dimensional stochastic dynamics.This methods is especially usefull in the context of molecular dynamics (MD) of proteins. Proteins and the obtaining of accurate stationary and kinetic behavior of protein is a crucial unsolved challenge which limits our understanding of}
behavior represents are common high-dimensional problem. all examples on proteins. But the adaptive sampling methods can be applied for any high-dimensional stochastic dynamics with the corresponding transition barriers.
Ever increasing demand for high data rate wireless transmissions with high spectral efficiency leads to utilization of communication systems with multiple transmit and receive antennas. Excellent quality of service represented with near-channel capacity error-rate performance can be achieved with iterative receiver structure composed of inner soft detection and outer soft-input soft-output decoding. Emerging wireless standards such as:~Wireless Local Area Network (W-LAN), Worldwide Interoperability for Microwave Access (WiMAX), $3^{rd}$ Generation Partnership Project Long Term Evolution (3GPP-LTE), etc are being constantly revised to provide higher data rates and better error-rate performance. Iterative receivers based on inner soft detection and outer decoding are promising solutions.

In this thesis, we propose to address issues of designing efficient physical layer receiver structure targeting its use in emerging wireless systems, including both downlink and uplink scenarios. It is our goal to develop performance-efficient wireless receiver with implementable hardware cost while achieving data throughputs in the order of hundreds MBits/sec. 

\section{Proteins and Molecular Dynamics (MD)}
\label{sec:MD}
check lorenzo chapter 1
Excellent error-rate performance in MIMO environment are made possible by employing sophisticated algorithms such as maximum \emph{a posteriori}~(MAP) detection techniques and outer channel decoding that provides error-correction in the presence of multiple access interference, burst channel fading, channel multi-paths, additive receiver noise, etc. An approximation of impractically complex optimal joint detection/decoding is achieved by iteratively improving the \emph{a posteriori} probabilities (APPs) of transmitted coded bits between inner soft detection and outer decoding. Inner detection is typically based on the simplification of exponentially complex maximum-likelihood~(ML) approach such as the sphere detection. 

\section{energy landscape }
energy landscape 
folding funnel
focker planck equation
\section{Sampling Problem}
check  lorenzo chapter 1

\section{TICA}
dimension reduction
check  lorenzo chapter 2

\section{vampnet intro}
vampnet intro

\section{MSM}
check  lorenzo chapter 2
mfpt

\section{Outline}

%\include{Introduction}
%\shipout\null

%1 -motivation, intro

% 2 - diffusion map - for dmdm for extasy 1

% 3 - theory - apper

% 4 software paper extasy1, paper extasy 2 part

% 5 applicatoon extasy paper2

% Conclusion

\include{Chapter1}    
%\shipout\null
%\include{Chapter2}    
%\shipout\null
\include{Chapter3}    
\include{Chapter32} 
%\shipout\null
\include{Chapter4}    
%\shipout\null
\include{Chapter5}    
%\shipout\null
%\include{Chapter6}    
%\shipout\null
\afterpage{\null\newpage}
\chapter{Conclusions}
\label{ch:conclude}
\emph{Adaptive sampling} increases our ability to predict conformational dynamics of high-dimensional stochastic systems or the general problem of high-dimensional stochastic systems. 
In theory adaptive sampling can increase sampling efficiency of a energy landscape by reducing the "redundant" samples. In practive the choice which sampling is "redundant" is a non-trivial and crucial step to increase the sampling efficiency of the sampling. This choice which of the sampling is "redundant" depends on the goal of the sampling and only approximations are currently possible. This reduces the effectivity of adaptive sampling to the theoretical maximum. 
In Chapter 3 this theoretical maximal efficiency of adaptive sampling is derived, first time develop by the author. It is shown that this maximum efficiency depends on the goal of the sampling, wether a maximal exploration is desired or a fast folding is desired. This maximum efficiency also depends on the protein and generally increases with the complexity and size of the protein. Adaptive sampling is more efficient in a complex transition region compare to a flat transitionregion. By extrapolating we can estimate that adaptive sampling can reach a speed up of at least 10-100 for larger proteins. By comparing the theoretical maximal efficiency we can show that additional improvements in adaptive sampling strategies are possible.

In Chapter 4 we could show that different adaptive sampling strategies are superior for different goals. The $cmicro$ strategy is better for exploration and the $cmacro$ startegy is better for crossing transition barriers such as folding. This is true across different protein, but the speed up achievable with adaptive sampling depends on the complexity of the proteins. Proteins with more complex timescales are expected to have a higher speed up with adaptive sampling compared to plain molecular dynamics. The stochasticity of molecular dynamics and sampling causes the time to solution to fluctuate significant, sometimes in order of 50\%. This means a single or few folding events, commonly caused by limited computational resources, has limited statistical significance when comparing adaptive sampling with plain molecular dynamics.

The improvements of the software package \emph{ExTASY} shown in Chapter 5 allow anyone to more eficiently to deploy adaptive sampling to sample any proteins. This packag ensures state-of-the-art scalability on High-Performance Computers and the modularity ensures the maintainalibity and user-friendly extensibility.

The application of \emph{ExTASY} in Chapter 5} shows that adaptive sampling reaches the promised speed ups for proteins up to size of 73 residues. For these proteins not only the folded state is recovered, but also conformational dynamics. Extending to larger proteins is only limited by computational resourced.

Together the results in this Disseration all contribute to better understanding of adaptive sampling. With better and better understood  adaptive sampling we are able to estimate the conformational dynamics of biomolecules for even longer timescales.  While the improvements shown are significant, additional improvements are necessary to reach even longer timescales. Adaptive sampling shows promise to contribute futher in reaching longer timescales, due to the significant gap between the practically reached speed up and the theoretically maximal speed up. Also the impacts of new question such as the asynchronous execution are unanswered.

While all the applications here are on biomolecules, the adaptive sampling works as well on the general problem of high-dimensional stochastic systems. For example, inorganic materials or deep learning where the loss function can represent the high-dimensional energy landscape.

Additional research of adaptive sampling could allow us to develop more effective adaptive sampling strategies or adaptive sampling strategies optimized for new goals, such as accurate mean first passage times. This dissertation is a starting point for further investigations.

   

\appendix

%\include{append-a}
%\appendix
%\addcontentsline{toc} {chapter}{\numberline {}Appendix}
%\include{append-a}
%\include{append-b}
%\addcontentsline{toc} {chapter}{\numberline {}Bibliography}{}
%\include{PhD_bib}

\bibliographystyle{naturemag}
\bibliography{thesis_main}
\end{document}
